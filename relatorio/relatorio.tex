\documentclass{article}

\usepackage{indentfirst}
\usepackage{graphicx}
\usepackage{listings}
\usepackage{fancyhdr}
\usepackage{hyperref}
\usepackage{amsmath}
\usepackage{amssymb}

\title{CIC202 - Programação Concorrente: Trabalho 1 \\
        \large \textbf{Sincronização de Processos:} A Caverna dos Sableyes}
\author{Guilherme da Rocha Cunha - 221030007}
\date{2024.1}

\begin{document}

\pagestyle{fancy}

\maketitle

\begin{figure}[ht]
        \centering
        \includegraphics[width=.5\textwidth]{imagens/as_vert_cor.jpg}
\end{figure}

\newpage

\fancyhead{}
\fancyhead[L]{\textbf{Sincronização de Processos:} A Caverna dos Sableyes}
\fancyfoot[C]{\thepage}


\renewcommand*\contentsname{Sumário}
\tableofcontents

\newpage

\section{Introdução}
O trabalho consiste na aplicação dos conhecimentos adquiridos na disciplina "CIC202 - Programação Concorrente". Nele é apresentado o problema de comunicação de processos atráves de uma memória compartilhada, "A Caverna dos Sableyes", e sua solução utilizando a biblioteca POSIX Pthereads na linguagem C.

\section{Formalização do Problema: A Caverna dos Sableyes}
Sableyes são uma espécie de pokemon de pele roxa, garras e dentes afiados, que se alimentam de pedras preciosas e moram no interior de cavernas.

\begin{figure}[ht]
        \centering
        \includegraphics[width = .25\textwidth]{imagens/sableye.png}
        \caption{Sableye}
\end{figure}

\section{Descrição do Algoritmo}

\section{Conclusão}

\section{Referências}

\end{document}